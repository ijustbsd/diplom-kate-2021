\section{Введение}

Хорошо известно (см., например, \cite{coddington}, \cite{hartman}, \cite{krasnoselskii}), что в случае
дифференциальных уравнений условие гладкости правых частей не обеспечивает существования решений задачи с
начальным значением на неограниченном справа интервале. Для существования таких решений в теории обыкновенных
дифференциальных уравнений широко применяется метод односторонней оценки (см., например, \cite{coddington}).
Для дифференциального уравнения в Гильбертовом пространстве $H$ вида
\begin{equation*}
    x' = f(t,х)
\end{equation*}
\noindent одна из простейших оценок такого типа может быть представлена
\begin{equation}
    \label{eq:f_condition}
    \langle f(t,x),x\rangle \leq a \norm{x}^{2} + b
\end{equation}
\noindent Чтобы получить существование решений на неограниченных интервалах в случае уравнений дробного порядка
обычно используют условие сублинейно растущей правой части (см. \cite{afanasova_nca}, \cite{afanasova_rm}, \cite{appell},
\cite{benedetti}, \cite{mainardi}, \cite{kamenskii_fpt}, \cite{kamenskii_aa}, \cite{kamenskii_fpta17}, \cite{kamenskii_fpta19},
\cite{kamenskii_m}). Очевидно, что это условие сильнее, чем (\ref{eq:f_condition}).

В данной работе мы доказываем существование слабых решений такого типа семилинейных дифференциальных включений дробного
порядка в сепарабельном Гильбертовом пространстве, при предположении, что линейная часть включения представлена неограниченным
отрицательно определённым оператором, а нелинейная часть удовлетворяет оценке (\ref{eq:f_condition}). Заметим, что
наша конструкция основана на следующем свойстве дробной производной Капуто функции $x(t)$ в Гильбертовом пространстве:

\begin{equation*}
    {}^{C} D_{0}^{q} \norm{x(t)}^{2} \leq \langle x(t), {}^{C} D_{0}^{q} x(t) \rangle,
\end{equation*}

\noindent которое изучалось в работах: 19, 20, 21.

Работа имеет следующую структуру: в следующем разделе мы представим некоторые предварительные сведения из
дробного нализа и теории уплотняющих многозначных отображений. В третьем разделе мы приводим результат априорной
оценки решений the initial value problem for a semilinear fractional differential nclusion in a Hilbert space under
assumption that the linear part of the inclusion is negatively defined and the multivalued nonlinearity satisfies an
one-sided estimate. Для доказательства этого результата воспользуемся приближёнными методами, онснованными на аппроксимации
Иосиды линейной части включения. В последнем разделе мы применим этот резульат для доказательства существования
слабого решения нашей начальной задачи на каждом конечном интервале и для проверки, что существующие решения ограничены
полуосями.

\section{Основные понятия}

\subsection{Дробная производная}

В этом разделе мы напомним некоторые понятия и определения, которые нам понадобятся в дальнейшем
(подробности можно найти в \cite{gorenflo}, \cite{kilbas}, \cite{podlubny}, \cite{zhou}).

Пусть $E$ Банахово пространство над полем вещественных чисел.

\begin{definition}
    Дробной производной Римана-Лиувилля порядка $q \in (0, 1)$ непрерывной функции $g:[0, a] \rightarrow E$ называется
    функция $D_{0}^{q}g$ определённая в следующем виде:
    $$D_{0}^{q}g(t) = \frac{1}{\Gamma(1 - q)}\frac{d}{dt}\int_{0}^{t}(t - s)^{-q}g(s)ds$$
    при условии, что правая часть этого равенства определена корректно.
\end{definition}

Здесь $\Gamma$ это гамма-функция Эйлера $$\Gamma(r) = \int_{0}^{\infty}s^{r-1}e^{-s}ds$$

\begin{definition}
    Дробной производной Капуто порядка $q \in (0, 1)$ непрерывнйо функции $g:[0, a] \rightarrow E$ называется функция
    ${}^CD_{0}^{q}g$ определённая следующим способом:
    $${}^CD_{0}^{q}g(t) = \Big(D^q(g(\cdot) - g(0))\Big)(t)$$
    при условии, что правая часть этого равенства определена корректно.
\end{definition}

\begin{definition}
    Функция
    $$E_{q,\beta}(z) = \sum_{n=0}^{\infty}\frac{z^n}{\Gamma(qn + \beta)}, \ \ \ q, \beta > 0, z \in \mathbb{C}$$
    называется функцией Миттаг-Леффлера.
\end{definition}

Функция Миттаг-Леффлера имеет следующее асимптотическое представление при $z \rightarrow \infty$:

\begin{equation}
    E_{q,\beta}(z) = 
    \begin{cases}
        \frac{1}{q}z^{\frac{1-\beta}{q}}e^{z^{\frac{1}{q}}} - \sum_{n=1}^{N-1}\frac{z^{-n}}{\Gamma(\beta-qn)} +
        O(|z|^{-N}), \ \ \ |argz| \leq \frac{1}{2} \pi q,\\
        - \sum_{n=1}^{N-1} \frac{z^{-n}}{\Gamma(\beta-qn)} + O(|z|^{-N}), \ \ \ |arg(-z)| \leq (1 - \frac{1}{2}q)\pi.
    \end{cases}
\end{equation}

Обозначим $E_{q,1}$ через $E_{q}$. Обратите внимание, что вторая из приведенных выше формул означает, что в случае
$z = \tau < 0$ и $0 < q < 1$ мы имеем
\begin{equation}
    E_{q}(\tau) \rightarrow 0 \ \text{при} \ \tau \rightarrow - \infty
\end{equation}

Заметим, что из отношений (см. \cite{wang}):

\begin{equation*}
    E_{q}(-z) = \int_{0}^{\infty} \xi(\theta)e^{-z\theta}d\theta
\end{equation*}

\noindent и

\begin{equation*}
    E_{q,q}(-z) = \int_{0}^{\infty} q\theta \xi(\theta)e^{-z\theta}d\theta,
\end{equation*}

\noindent где

\begin{equation}
    \xi_{q}(\theta) = \frac{1}{q} \theta^{-1-\frac{1}{q}}\Psi_{q}(\theta^{\frac{-1}{q}}),
\end{equation}

\begin{equation}
    \Psi_{q}(\theta) = \frac{1}{\pi} \sum_{n=1}^{\infty} (-1)^{n-1} \theta^{-qn-1} \frac{\Gamma(nq+1)}{n!} \sin(n \pi q), \ 
    \theta \in \mathbb{R}_{+},
\end{equation}

\noindent следует, что

\begin{equation}
    E_{q}(\tau) > 0, E_{q,q}(\tau) > 0 \ \text{для} \ \tau < 0.
\end{equation}

\begin{remark}
    (См. \cite{zhang}) $\int_{0}^{\infty} \theta \xi_{q}(\theta) d\theta = \frac{1}{\Gamma(q+1)}$,
    $\int_{0}^{\infty} \xi_{q}(\theta) d\theta = 1$, $\xi_{q}(\theta) \geq 0$.
\end{remark}

Рассмотрим скалярное уравнение вида

\begin{equation}
    \label{eq:dxt}
    {}^{C}D^{q} x(t) = \lambda x(t) + f(t), \ \ \ t \in [0, T]
\end{equation}

\noindent с начальным условием

\begin{equation}
    \label{eq:dxt_x0}
    x(0) = x_{0},
\end{equation}

\noindent где $\lambda \ in \mathbb{R}, f : [0, T] \rightarrow \mathbb{R}$ - непрерывная функция. Решением этого уравнения мы
будем называть непрерывную функцию $x : [0, T] \rightarrow \mathbb{R}$, удовлетворяющую условию (\ref{eq:dxt_x0}),
дробная производная ${}^{C}D^{q}x$ которой также непрерывная и удовлетворяет выражению (\ref{eq:dxt}). Известно, что единственное
решение данного уравнения имеет вид

\begin{equation}
    x(t) = E_{q}(\lambda t^{q})x_{0} + \int_{0}^{t} (t-s)^{q-1} E_{q,q}(\lambda(t-s)^{q})f(s)ds.
\end{equation}

Нам понадобится следующее вспомогательное утверждение, являющееся аналогом известной
леммы Гронуолла — Беллмана об интегральных неравенствах \cite{kamenskii_aa}.

\begin{lemma}
    Пусть ограниченная измеримая функция $\omega: [0, T] \rightarrow \mathbb{R}$ удовлетворяет интегральному неравенству
    \begin{equation}
        \omega(t) \leq E_{q} (-\eta t^{q}) \omega(0) + \int_{0}^{t}(t-s)^{q-1} E_{q,q}(-\eta(t-s)^{q})\Big(K + m\omega(s)\Big)ds
    \end{equation}
    где $K \geq 0, \ 0 < m < \eta$. Тогда
    \begin{equation*}
        \omega(t) \leq E_{q} \Big((-\eta+m)t^{q}\Big) \omega(0) + K \int_{0}^{t} (t-s)^{q} E_{q,q} \Big((-\eta+m)(t-s)^{q}\Big)ds.
    \end{equation*}
\end{lemma}

\subsection{Меры некомпактности}

\clearpage

\section{Априорные оценки решений}

\clearpage

\section{Существование результата}

\clearpage

\section{Ограниченность решений на полуоси}

\clearpage

\addcontentsline{toc}{section}{Список литературы}

\nocite{*}

\printbibliography{}
